\section*{Present State of Knowledge}

%% From the CFP: Provide a brief overview indicating the nature and
%% importance of the project; place its importance within the context
%% of general knowledge in your field and indicate possible practical
%% significance, if applicable. Include appropriate references from
%% relevant literature to help place the project in context and
%% further indicate your familiarity with the field.

\noindent Programming languages are fundamental to the
permeation of computing into everyday applications. 
%
Research on such languages is geared towards making them effective
tools in this domain.
%
Such work focuses on the two complementary aspects of
\emph{expressivity} and \emph{verifiability}.
%
Expressivity give substance to the sentiment that languages
should make it easy to translate conceived solutions to computational
problems into actual, working programs.
%
Verifiability dictates that the languages we design should make it
easy to ascertain that the programs we write in fact realize the
intended computations.
%
These two concerns have mediated the transition from low-level
assembler languages to the high-level languages that underlie the
explosive growth of sophisticated and usually functional software.
%
Modern-day applications have raised a new challenge for the notion
of expressivity vis-a-vis programming languages: they must support
primitives that are easy for experts in specialized domains to
understand and use.
%
\emph{Language extensibility} has been proposed and
investigated as a means for meeting this challenge.
%
While this idea has been successful in enhancing expressivity, it
raises new questions for verifiability. 
%
The work we propose fits under a broader project that the PIs are
undertaking to understand these questions and to build a platform for  
addressing them.
%
We discuss the interactions between expressivity, extensibility and
verifiability in more detail below to provide a context for the
specific work we will undertake.  

\medskip
\noindent {\bf Language expressivity and extensibility.} Expressivity
was considered in the past to be an attribute of one properly designed
language.   
%
However, this view has been challenged by the realization that the
meaning of expressivity depends crucially on the domain of
computation. 
%
One reaction to this situation has been to persist with the idea of
one ``general purpose'' language but to design specific extensions to
it that are targeted to providing particular missing features.
%
There are several examples of such efforts, ranging from ones that add 
capabilities for supporting database queries or concurrent
programming~\cite{price2001sqlj,frigo89pldi}, to those that add new
analyses that ensure security in information flow or unwanted
interference does not occur in concurrent programs that work over the
same data~\cite{myers99popl,georges2025popl}. 
%
Language
extensibility~\cite{ekman07oopsla,erdweg11oopsla,tobin2011languages,zenger2005fool}
has been investigated as a {\it different} approach to solve this
problem.
%
The central idea underlying this approach is that a language is to be
viewed as the composition of a \emph{host language} that provides a
core set of programming features with components chosen from an
\emph{open library of independent extensions} developed around the
host language that contribute additional desired features.
%
This line of research brings a \emph{foundational} perspective to
language extensibility: it asks questions about the properties a
library should possess to make compositions work
and to give programmers a seamless view of a language that ``under
the  covers'' may actually be the result of combining different
components from a library. 

\medskip
\noindent {\bf Language extensibility and modularity.} A fundamental
requirement of the library view is that language extensions must be
{\it composable}, \ie, it should be possible to use an extension in
combination with \emph{any collection} of other extensions from the
library to produce a working and well-behaved language.
%
For this to be possible, the library must support the ability for an 
extension to check its compatibility \emph{without specific knowledge
of the other extensions presently in the library}. 
%
We refer to this feature as the {\it modularity property} for the
framework.
%
One problem that must be solved in this context pertains to
language specifications: the vocabulary, constructs and associated
analyses introduced by different extensions might overlap, potentially
resulting in an ill-defined composition.
%
This issue has been addressed by several research groups,
including the one led by one of the PIs.
%
For example, criteria have been enunciated for ensuring that the
composition of extensions with overlapping syntax will be syntactically
well-defined~\cite{schwerdfeger09pldi}, and these have
been complemented by concrete methods for checking that a given
extension satisfies the formal guidelines.
%
Similar work has been done at the level of ensuring that the analyses
that an extension associates with the constructs it defines are
compatible with what is required by the library and, conversely, that
any new analyses it introduces will fit with the library
structure~\cite{kaminski12sle}. 
%
In short, modularity in language specification in the context of the
library view of language extensibility is largely a solved problem
today. 

\medskip
\noindent {\bf Language extensionality and verifiability.} A key
aspect of a programming language that facilitates reasoning about
program behavior are its {\it meta-theoretic properties}.
%
These are properties that are guarantee to hold of {\it any} program
just because they are written in that language.
%
%% One common example of a meta-theoretic property is that which says a
%% well-formed program in a typed language will never encounter a type
%% error when it is run: this property, in turn, allows us to conclude
%% that a program that has passed the checks in a compiler will not stall
%% during execution.
%
The development of methods and mechanical theorem-proving systems that
support the effective verification of such properties is a cornerstone
of modern research in the area (\eg, see~\cite{pierce2026sf}).
%
However, almost all this work has been focused on a situation where
a language is provided as a monolithic whole, a condition that
\emph{does not} obtain in the library-based view of extensibility that
underlies this proposal. 
%
The PIs, who have individual expertise in extensible languages and
formal treatments of meta-theory, have started a collaboration in
recent years to develop an approach to proving meta-theoretic
properties that will work for extensible languages.
%
As with language specifications, an important requirement is that the
approach must be modular: it should be possible for individual
extensions to contribute to 
proofs of global properties for the constructs they add to a composed  
language and the framework must also support a means to prove a
property articulated by an extension concerning an analysis that it
has introduced.
%
The PIs have developed a methodology for realizing these requirements
and have established its theoretical underpinnings in recent
work~\cite{michaelson2025toplas}.
%
This award request is for research that will explore certain practical
aspects of the scheme that seems important to securing NSF support for
a more ambitious investigation.  
