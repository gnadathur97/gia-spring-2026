\section*{Plan of Work}

%% From the CFP:
%%   o Description and a timeline of the work you propose to undertake 
%%   o History of the project
%%   o Elements of the project you expect to complete during the grant
%%     period 
%%   o Materials and methods to be used
%%   o Hypotheses to be tested or the specific questions to be
%%     addressed, and the approaches you will take to reach your goals
%%     (where appropriate) 
%%   o Use sufficient detail to permit a critical evaluation of the
%%     project's likelihood of success and address potential
%%     significance of the work 

The conventional role of metatheory for programming languages it to
provide an assurance that analyses that are carried out statically
over programs are accurate predictors of their execution behavior.
%
A canonical example of this kind arises in 
typechecking.
%
This is an analysis 
that determines a program's acceptability
before it is run.
%
However, its purpose is to ensure that a program that is deemed to be
well-formed will not stall during execution because an operation is
attempted on data to which it cannot be meaningfully
applied.
%
Clearly, the static analysis is not directly linked to the dynamic
behavior.
%
Rather, the linkage must be established through a reasoning process.
%
That process must have as input a specification of the execution
process and it must reason based on it to show that
``type-correctness'' is preserved over computational steps.

From what is said above, it becomes clear that metatheory involves the
interaction of three distinct components.
%
At the outset, there are the language specifications, which encompass
those describing analyses (such as type-checking) and those specifying
behavior (such as the execution steps).
%
Next, there is an articulation of properties that are intended to link
the analyses to program behavior.
%
Finally, it is necessary to provide proofs of the properties, which
must be based on language specifications and, of course, a knowledge
of what is being asserted.
%
The difficulties in the library-based view of extensible languages can
now be seen to be of two kinds. 
%
First, when an extension describes an analysis it must do this without
knowledge of the constructs added by other extensions in the library.
%
Second, we must decide how to distribute the construction of proofs
across different components and we have to deal with the fact that any
component that participates in the task will have only partial
knowledge of the language constructs and associated specifications.

In their recent collaborative work, the PIs have pioneered an effort
to come to grips with the metatheory problem for extensible
languages~\cite{michaelson2025toplas}. 
%
A key idea contributed by their work is that the host language can
provide a medium through which extensions in the library can
``understand'' each other.
%
Specifically, when an extension adds a new construct to a
syntactic category identified by the host language---which is the only
way it can share linguistic features with other extensions---then it
must describe a \emph{projection} onto host language syntax that other
extensions can use as an image for its constructs.
%
Further, the extension should ensure that a specific relationship
holds between the semantic attributes of its constructs and their
projections.
%
These relationships, which are called \emph{projection constraints},
are specified by the host language and amounts to it setting the
parameters for participation in the language library.
%
Extensions must establish that they satisfy the projection constraints
as their ``right of passage.''

Using the structure described above, the PIs have developed a
methodology for modularizing proofs of language properties.
%
Properties have been distinguished into two kinds called
\emph{foundational} and \emph{auxiliary} in this work.
%
Foundational properties are ones that are articulated by the host
language and their proofs are contributed to by each component in the
library. 
%
An auxiliary property is oriented around an analysis introduced by an
extension and it is therefore a property about with other components
have no knowledge. 
%
Thus, the proof of an auxiliary property must be constructed
entirely within the extension that introduces it but, of course, it
may use knowledge of the host language and of the projection
constraints.  
%
The work has developed a methodology for constructing such proofs that
has been shown to be sound by using the concrete structure of a logic
and proof system originating from the past research of one of the PIs.
%
Moreover, the overall framework has been implemented in a system
called \emph{Extensibella}, thereby providing a vehicle for
experimentation with the methodology. 

While the theoretical underpinnings for the modular reasoning
framework are now established, much remains to be
done towards demonstrating its practical viability.
%
This project will make a contribution towards this end.
%
It will specifically target the two following aspects.
%
\begin{enumerate}
\item The examples that have been used in motivating and developing
  the framework have relied on a conventional ``syntactic'' approach
  to specifying the semantics of programming languages.
  %
  There is a growing interest in an alternative ``logical''
  approach~\cite{timany2024jacm} that adds considerable power to
  metatheoretical analysis and is increasingly being used in
  applications. 
  %
  It is our assessment at the outset that the method of defining
  semantics and articulating properties is orthogonal to the issue of
  modularizing proofs; if anything, we believe the new methods may
  dovetail into the way we have proposed to prove auxiliary
  properties.
  %
  However, this is a hypothesis that needs validation, something that
  we will endeavor to provide.

\item Our approach to treating interactions between extensions
  involves having the host language impose structure on the library
  through projection constraints.
  %
  At a formal level, this accords well with a motto in programming
  language work that dependability is the result of discipline.
  %
  However, there are many empirical questions concerning the
  connections between the richness of the host language, the strength
  of projection constraints, the flexibility of extensions, and the
  depth of the properties that can be proved that need to be further
  explored. 
  %
  Using examples from Van Wyk's extensibility libraries,
  we will conduct experiments towards understanding the
  choices in these realms better.
\end{enumerate}
%
The work in both cases will be carried out by developing an
understanding of proof methods, identifying language libraries that
will provide insights into the underlying questions, constructing
models and proofs, and analyzing aspects such as the extensions that
can be described and the properties that can be expressed and proved
within the framework. 
%
The \emph{Extensibella} system will be used in the experiments.
%
The primary effort will be that of a graduate student, who will need
to have an adequate theoretical background and an interest in 
programming languages research.
%
Both PIs will oversee the effort.

The questions the proposal will investigate
were raised about an NSF proposal for which funding was recently
declined despite a competitive rating.
%
We have responded to these questions in a different way in a
recent resubmission.
%
The successful completion of this project will enable a direct
response to what we see as the main obstacle to NSF funding this 
thrust in our research.
